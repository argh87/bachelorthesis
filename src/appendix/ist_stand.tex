\chapter{Bestehender Quellcode}
\markboth{A. Bestehender Quellcode}{}
\label{ch:Quellcode}

Im Folgenden wird der vor der Bearbeitung und Optimierung bestehende Quellcode aufgelistet, der im direkten Zusammenhang mit der Transaktionserfassung steht. Der vollständige Quellcode befindet sich im Git-Repository von WBS Alarm als Release\footnote{\url{https://github.com/argh87/de.hs-fulda.et.it.project/releases/tag/1.0}} mit der Version 1.0.

\begin{lstlisting}[caption={Schnittstelle der Aktion zur Erfassung einer Transaktion.}, label={lst:AddTransaktionAction}]
package de.hsfulda.et.wbs.action.transaktion;
import de.hsfulda.et.wbs.core.WbsUser;
import de.hsfulda.et.wbs.core.data.TransaktionData;
import de.hsfulda.et.wbs.core.dto.TransaktionDto;
public interface AddTransaktionAction {
  TransaktionData perform(WbsUser user, TransaktionDto dto);
}
\end{lstlisting}

\begin{lstlisting}[caption={Implementierung der Aktion zum Erfassen der Transaktion.}, label={lst:AddTransaktionActionImpl}]
package de.hsfulda.et.wbs.action.transaktion.impl;
import de.hsfulda.et.wbs.action.transaktion.AddTransaktionAction;
import de.hsfulda.et.wbs.core.WbsUser;
import de.hsfulda.et.wbs.core.data.TransaktionData;
import de.hsfulda.et.wbs.core.dto.TransaktionDto;
import de.hsfulda.et.wbs.entity.Transaktion;
import de.hsfulda.et.wbs.repository.TransaktionRepository;
import de.hsfulda.et.wbs.service.AccessService;
import org.springframework.stereotype.Component;
import javax.transaction.Transactional;
@Transactional
@Component
public class AddTransaktionActionImpl implements AddTransaktionAction {
  private final TransaktionValidaton validation;
  private final TransaktionExecution execution;
  private final TransaktionRepository transaktionen;
  private final TransaktionMailService transaktionMailService;
  private final AccessService accessService;
  public AddTransaktionActionImpl(TransaktionValidaton validation, TransaktionExecution execution,
      TransaktionRepository transaktionen, TransaktionMailService transaktionMailService,
      AccessService accessService) {
    this.validation = validation;
    this.execution = execution;
    this.transaktionen = transaktionen;
    this.transaktionMailService = transaktionMailService;
    this.accessService = accessService;
  }
  @Override
  public TransaktionData perform(WbsUser user, TransaktionDto dto) {
    return accessService.hasAccessOnTransaktion(user, dto, () -> {
      validation.validateTransaktionDto(dto);
      Transaktion transaktion = execution.createTransaktion(user, dto);
      transaktionMailService.sendMail(user, dto);
      return transaktionen.save(transaktion);
    });
  }
}
\end{lstlisting}

\begin{lstlisting}[caption={Validierung der übergebenen Daten einer Transaktion.}, label={lst:TransaktionValidaton}]
package de.hsfulda.et.wbs.action.transaktion.impl;
import de.hsfulda.et.wbs.core.data.BestandData;
import de.hsfulda.et.wbs.core.data.GroesseData;
import de.hsfulda.et.wbs.core.data.KategorieData;
import de.hsfulda.et.wbs.core.data.ZielortData;
import de.hsfulda.et.wbs.core.dto.PositionDto;
import de.hsfulda.et.wbs.core.dto.TransaktionDto;
import de.hsfulda.et.wbs.core.exception.ResourceNotFoundException;
import de.hsfulda.et.wbs.core.exception.TransaktionValidationException;
import org.springframework.stereotype.Component;
import java.util.Collections;
import java.util.List;
import java.util.Objects;
import java.util.Set;
import java.util.stream.Collectors;
@Component
class TransaktionValidaton {
  private final TransaktionContext context;
  TransaktionValidaton(TransaktionContext context) {
    this.context = context;
  }
  /**
   * Die Transaktion wird an dieser Stelle validiert. Dabei wird geprüft ob die beiden Zielorte dem gleichen Träger
   * zugeordnet sind. Desweiteren wird geprüft, ob die Zielorte und die Größen existieren. Dabei ist zu beachten,
   * dass Zielorte und Größen aktiv sind. Auf inaktive darf nicht gebucht werden. Hierbei wird ein 404 ausgelöst über
   * die {@link ResourceNotFoundException}. Die Erfassung der Bestände muss zudem abgeschlossen sein. Dazu wird
   * geprüft, ob genug Bestand im ausgehenden Zielort vorhanden ist. Es wird zudem geprüft, ob mindestens eine
   * Position ind er Transaktion angegeben wurde. Gleiche Größen (und die dazugehörige Kategorie) darf nicht in
   * verschiendenen Positionen auftauchen.
   *
   * @param dto Übergebene Daten einer Transaktion.
   */
  void validateTransaktionDto(TransaktionDto dto) {
    ZielortData vonZielort = context.getZielortData(dto.getVon());
    ZielortData nachZielort = context.getZielortData(dto.getNach());
    if (!hasSameTraeger(vonZielort, nachZielort)) {
      throw new TransaktionValidationException("Die Zielorte {0} und {1} besitzen nicht den gleichen Träger.",
          vonZielort.getName(), nachZielort.getName());
    }
    // Ausgehender Zielort erfasst?
    if (!vonZielort.isErfasst()) {
      throw new TransaktionValidationException(
          "Der Zielort \"{0}\" muss in der Erfassung der Bestände abgeschlossen sein. " +
              "Bitte wenden Sie sich an ihren Systembetreuer.", vonZielort.getName());
    }
    // Zielort erfasst?
    if (!nachZielort.isErfasst()) {
      throw new TransaktionValidationException(
          "Der Zielort \"{0}\" muss in der Erfassung der Bestände abgeschlossen sein. " +
              "Bitte wenden Sie sich an ihren Systembetreuer.", nachZielort.getName());
    }
    // Zielort ist nicht der Wareneingang
    if (nachZielort.isEingang()) {
      throw new TransaktionValidationException(
          "Der Zielort \"{0}\" ist als Wareneingang definiert. Dieser kann nicht als Zielort angegeben " +
              "werden.", nachZielort.getName());
    }
    // Es wurde mindestens eine Position angegeben.
    List<PositionDto> positions = dto.getPositions();
    if (positions.isEmpty()) {
      throw new TransaktionValidationException("Es muss mindestens eine Position angegeben werden.");
    }
    Set<Long> duplicates = getDuplicateGroesse(positions);
    if (!duplicates.isEmpty()) {
      throw new TransaktionValidationException(
          "Die Größe mit der/den ID/s {0} wurde in den Positionen doppelt angegeben.", duplicates);
    }
    // Der Wareneingang in den Beständen nicht aktualisiert.
    if (!vonZielort.isEingang()) {
      // Existiert genug Bestand für eine Position
      positions.forEach(p -> {
        BestandData bestand = context.getBestandData(vonZielort.getId(), p.getGroesse());
        GroesseData groesse = context.getGroesseData(p.getGroesse());

        if (bestand.getAnzahl() < p.getAnzahl()) {
          KategorieData kategorie = groesse.getKategorie();
          throw new TransaktionValidationException(
              "Die Position mit Kategorie \"{0}\", Größe \"{1}\", Anzahl {2} übersteigt den " +
                  "Bestand vom ausgehenden Zielort {3}.", kategorie.getName(), groesse.getName(),
              p.getAnzahl(), vonZielort.getName());
        }
      });
    }
  }
  private static Set<Long> getDuplicateGroesse(List<PositionDto> positions) {
    List<Long> numbers = positions.stream()
        .map(PositionDto::getGroesse)
        .collect(Collectors.toList());
    return numbers.stream()
        .filter(i -> Collections.frequency(numbers, i) > 1)
        .collect(Collectors.toSet());
  }
  private boolean hasSameTraeger(ZielortData von, ZielortData nach) {
    return Objects.equals(von.getTraeger(), nach.getTraeger());
  }
}
\end{lstlisting}

\begin{lstlisting}[caption={Speicherung einer neuen Transaktion.}, label={lst:TransaktionExecution}]
package de.hsfulda.et.wbs.action.transaktion.impl;
import de.hsfulda.et.wbs.core.WbsUser;
import de.hsfulda.et.wbs.core.data.BestandData;
import de.hsfulda.et.wbs.core.dto.PositionDto;
import de.hsfulda.et.wbs.core.dto.TransaktionDto;
import de.hsfulda.et.wbs.core.exception.ResourceNotFoundException;
import de.hsfulda.et.wbs.entity.Bestand;
import de.hsfulda.et.wbs.entity.Position;
import de.hsfulda.et.wbs.entity.Transaktion;
import de.hsfulda.et.wbs.entity.Zielort;
import org.springframework.stereotype.Component;
import java.time.LocalDateTime;
import java.util.Optional;
@Component
class TransaktionExecution {
  private final TransaktionContext context;
  private final CreateBestandForTransaktion createBestand;
  private final UpdateBestandForTransaktion updateBestand;
  TransaktionExecution(TransaktionContext context, CreateBestandForTransaktion createBestand,
             UpdateBestandForTransaktion updateBestand) {
    this.context = context;
    this.createBestand = createBestand;
    this.updateBestand = updateBestand;
  }
  Transaktion createTransaktion(WbsUser user, TransaktionDto dto) {
    Zielort vonZielort = context.getZielort(dto.getVon());
    Transaktion.TransaktionBuilder builder = Transaktion.builder()
        .setBenutzer(context.getBenutzer(user))
        .setDatum(LocalDateTime.now())
        .setVon(vonZielort)
        .setNach(context.getZielort(dto.getNach()));
    dto.getPositions()
        .forEach(p -> {
          builder.addPosition(Position.builder()
              .setAnzahl(p.getAnzahl())
              .setGroesse(context.getGroesse(p.getGroesse()))
              .build());

          // Der Wareneingang soll nicht gebucht werden.
          if (!vonZielort.isEingang()) {
            updateVonBestand(p, dto.getVon());
          }
          updateNachBestand(p, dto.getNach());
        });

    return builder.build();
  }
  /**
   * Ermittelt den Bestand vom Ausgangsort und aktualisiert den Bestand indem die Anzahl der Position abgezogen wird.
   * Wenn Bestand nicht existiert, wird ein Fehler geworfen.
   *
   * @param position Position, die verarbeitet wird.
   * @param von      Zielort von dem eine Bestandteil an einen anderen Zielort übergeben werden soll.
   */
  private void updateVonBestand(PositionDto position, Long von) {
    Optional<Bestand> vonBestand = context.getBestand(von, position.getGroesse());
    if (!vonBestand.isPresent()) {
      throw new ResourceNotFoundException(
          "Es gibt keinen Bestand von dem die eine Position nicht abgebucht werden kann.");
    }

    substractAnzahl(position, vonBestand.get());
  }
  /**
   * Aktualisieren der Anzahl indem die in der Position übergebene Anzahl abgezogen wird.
   *
   * @param position Position, die verarbeitet wird.
   * @param bestand  Zu aktualisierender Bestand.
   */
  private void substractAnzahl(PositionDto position, Bestand bestand) {
    updateBestand.perform(bestand.getId(), () -> bestand.getAnzahl() - position.getAnzahl());
  }
  /**
   * Aktualisiert den Bestand des Zielorts an den eine Position abgegeben wird. Existiert noch kein Bestand beim
   * Zielort wird dieser erstellt und mit keinem Bestand initialisiert. danach wird die Anzahl der aktuellen
   * Position dem Bestand hinzugerechnet.
   *
   * @param position Position, die verarbeitet wird.
   * @param nach     Zielort, an den die Position geht.
   */
  private void updateNachBestand(PositionDto position, Long nach) {
    Bestand nachBestand = getNachBestand(position, nach);
    addAnzahl(position, nachBestand);
  }
  /**
   * Ermittelt den Bestand anhan des Zielorts und der Position. Sollte der Bestand nciht existieren wird er angelegt.
   *
   * @param position Position, die verarbeitet wird.
   * @param nach     Zielort, an den die Position geht.
   * @return Persitierter Bestand.
   */
  private Bestand getNachBestand(PositionDto position, Long nach) {
    Optional<Bestand> nachBestand = context.getBestand(nach, position.getGroesse());
    if (!nachBestand.isPresent()) {
      BestandData created = createBestand.perform(nach, BestandCreateDtoImpl.of(position));
      return context.getBestand(created.getId());
    }
    return nachBestand.get();
  }
  /**
   * Aktualisieren der Anzahl indem die in der Position übergebene Anzahl hinzugefügt wird.
   *
   * @param position Position, die verarbeitet wird.
   * @param bestand  Zu aktualisierender Bestand.
   */
  private void addAnzahl(PositionDto position, Bestand bestand) {
    updateBestand.perform(bestand.getId(), () -> bestand.getAnzahl() + position.getAnzahl());
  }
}
\end{lstlisting}

\begin{lstlisting}[caption={Kontext und Abstraktion der Daten für die Erfassung von Transaktionen.}, label={lst:TransaktionContext}]
package de.hsfulda.et.wbs.action.transaktion.impl;
import de.hsfulda.et.wbs.core.WbsUser;
import de.hsfulda.et.wbs.core.data.BestandData;
import de.hsfulda.et.wbs.core.data.GroesseData;
import de.hsfulda.et.wbs.core.data.ZielortData;
import de.hsfulda.et.wbs.core.exception.ResourceNotFoundException;
import de.hsfulda.et.wbs.core.exception.TransaktionValidationException;
import de.hsfulda.et.wbs.entity.Benutzer;
import de.hsfulda.et.wbs.entity.Bestand;
import de.hsfulda.et.wbs.entity.Groesse;
import de.hsfulda.et.wbs.entity.Zielort;
import de.hsfulda.et.wbs.repository.BenutzerRepository;
import de.hsfulda.et.wbs.repository.BestandRepository;
import de.hsfulda.et.wbs.repository.GroesseRepository;
import de.hsfulda.et.wbs.repository.ZielortRepository;
import org.springframework.stereotype.Component;
import java.util.Optional;
@Component
class TransaktionContext {
  private final BenutzerRepository benutzer;
  private final GroesseRepository groessen;
  private final BestandRepository bestaende;
  private final ZielortRepository zielorte;
  private TransaktionContext(BenutzerRepository benutzer, GroesseRepository groessen, BestandRepository bestaende,
      ZielortRepository zielorte) {
    this.benutzer = benutzer;
    this.groessen = groessen;
    this.bestaende = bestaende;
    this.zielorte = zielorte;
  }
  Benutzer getBenutzer(WbsUser user) {
    Optional<Benutzer> managed = benutzer.findById(user.getId());
    return managed.orElseThrow(
        () -> new ResourceNotFoundException("Benutzer mit ID {0} nicht gefunden.", user.getId()));
  }
  ZielortData getZielortData(Long id) {
    Optional<ZielortData> managed = findZielortByIdAndAktivIsTrue(id);
    return managed.orElseThrow(() -> new ResourceNotFoundException("Zielort mit ID {0} nicht gefunden.", id));
  }
  Zielort getZielort(Long id) {
    Optional<Zielort> managed = zielorte.findById(id);
    return managed.orElseThrow(() -> new ResourceNotFoundException("Zielort mit ID {0} nicht gefunden.", id));
  }
  private Optional<ZielortData> findZielortByIdAndAktivIsTrue(Long id) {
    return zielorte.findByIdAndAktivIsTrue(id);
  }
  GroesseData getGroesseData(Long id) {
    Optional<GroesseData> managed = findGroesseByIdAndAktivIsTrue(id);
    return managed.orElseThrow(() -> new ResourceNotFoundException("Groesse mit ID {0} nicht gefunden.", id));
  }
  Groesse getGroesse(Long id) {
    Optional<Groesse> managed = groessen.findById(id);
    return managed.orElseThrow(() -> new ResourceNotFoundException("Groesse mit ID {0} nicht gefunden.", id));
  }
  private Optional<GroesseData> findGroesseByIdAndAktivIsTrue(Long id) {
    return groessen.findByIdAndAktivIsTrue(id);
  }
  BestandData getBestandData(Long zielortId, Long groesseId) {
    Optional<BestandData> managed = findBestandByZielortIdAndGroesseId(zielortId, groesseId);
    return managed.orElseThrow(
        () -> new TransaktionValidationException("Für den Zielort {0} wurde kein Bestand gefunden (Größe {1}).",
            zielortId, groesseId));
  }
  Optional<Bestand> getBestand(Long zielortId, Long groesseId) {
    return bestaende.findByZielortIdAndGroesseId(zielortId, groesseId);
  }
  Bestand getBestand(Long bestandId) {
    Optional<Bestand> managed = bestaende.findById(bestandId);
    return managed.orElseThrow(
        () -> new ResourceNotFoundException("Bestand mit ID {0} nicht gefunden.", bestandId));
  }
  private Optional<BestandData> findBestandByZielortIdAndGroesseId(Long zielortId, Long groesseId) {
    return bestaende.findByZielortIdAndGroesseIdAsData(zielortId, groesseId);
  }
}
\end{lstlisting}

\begin{lstlisting}[caption={Repository in Stil von Spring für die Verwaltung von Transaktionen.}, label={lst:TransaktionRepository}]
package de.hsfulda.et.wbs.repository;
import de.hsfulda.et.wbs.core.data.TransaktionData;
import de.hsfulda.et.wbs.entity.Transaktion;
import org.springframework.data.domain.Page;
import org.springframework.data.domain.Pageable;
import org.springframework.data.jpa.repository.Query;
import org.springframework.data.repository.CrudRepository;
import org.springframework.data.repository.query.Param;
import java.util.List;
import java.util.Optional;
public interface TransaktionRepository extends CrudRepository<Transaktion, Long> {
  @Query("SELECT t FROM Transaktion t")
  List<TransaktionData> findAllAsData();
  @Query("SELECT t FROM Transaktion t where t.id = :id")
  Optional<TransaktionData> findByIdAsData(@Param("id") Long id);
  @Query("SELECT t FROM Transaktion t JOIN t.von v JOIN v.traeger s where s.id = :traegerId ORDER BY t.datum DESC")
  Page<TransaktionData> findAllAsDataByTraegerId(@Param("traegerId") Long traegerId, Pageable pageable);
}
\end{lstlisting}

\begin{lstlisting}[caption={Repository im Stil von \textit{Spring Boot Data} für die Zugriffsprüfung.}, label={lst:AccessRepository}]
package de.hsfulda.et.wbs.repository;
import de.hsfulda.et.wbs.entity.Traeger;
import org.springframework.data.jpa.repository.Query;
import org.springframework.data.repository.Repository;
import org.springframework.data.repository.query.Param;
public interface AccessRepository extends Repository<Traeger, Long> {
  // ...
  @Query("SELECT COUNT(t) FROM Traeger t JOIN t.benutzer b JOIN t.zielorte z where b.id = :userId and z.id = :zielortId")
  Long findTraegerByUserAndZielortId(@Param("userId") Long userID, @Param("zielortId") Long zielortId);
  @Query("SELECT COUNT(t) FROM Traeger t JOIN t.benutzer b JOIN t.kategorien k JOIN k.groessen g where b.id = :userId and g.id = :groesseId")
  Long findTraegerByUserAndGroesseId(@Param("userId") Long userID, @Param("groesseId") Long groesseId);
  // ...
}
\end{lstlisting}

\begin{lstlisting}[caption={Serviceklasse für die Zugriffsprüfung.}, label={lst:AccessService}]
package de.hsfulda.et.wbs.service;
import de.hsfulda.et.wbs.core.WbsUser;
import de.hsfulda.et.wbs.core.dto.TransaktionDto;
import de.hsfulda.et.wbs.core.exception.ResourceNotFoundException;
import de.hsfulda.et.wbs.repository.AccessRepository;
import org.springframework.stereotype.Service;
import java.util.function.Supplier;
@Service
public class AccessService {
  private final AccessRepository repo;
  public AccessService(AccessRepository repo) {
    this.repo = repo;
  }
  //...
  public <T> T hasAccessOnTransaktion(WbsUser user, Long transaktionId, final Supplier<T> supplier) {
    Long counts = repo.findTraegerByUserAndTransaktionId(user.getId(), transaktionId);
    return evaluteCount(counts, transaktionId, "Transaktion", supplier);
  }
  public <T> T hasAccessOnTransaktion(WbsUser user, TransaktionDto dto, final Supplier<T> supplier) {
    checkCount(repo.findTraegerByUserAndZielortId(user.getId(), dto.getVon()), dto.getVon(), "Zielort");
    checkCount(repo.findTraegerByUserAndZielortId(user.getId(), dto.getNach()), dto.getNach(), "Zielort");
    dto.getPositions()
        .forEach(p -> {
          checkCount(repo.findTraegerByUserAndGroesseId(user.getId(), p.getGroesse()), p.getGroesse(), "Größe");
        });
    return supplier.get();
  }
  private void checkCount(final Long counts, final Long id, final String resource) {
    if (counts == 0) {
      throw new ResourceNotFoundException("{1} mit ID {0} nicht gefunden.", id, resource);
    }
  }
  private <T> T evaluteCount(final Long counts, final Long id, final String resource, final Supplier<T> supplier) {
    checkCount(counts, id, resource);
    return supplier.get();
  }
}
\end{lstlisting}

\begin{lstlisting}[caption={Serviceklasse für das Versenden von E-Mails.}, label={lst:MailService}]
package de.hsfulda.et.wbs.service;
import de.hsfulda.et.wbs.core.Mail;
import de.hsfulda.et.wbs.core.exception.MailConnectionException;
import de.hsfulda.et.wbs.core.exception.MailDeliveryException;
import org.slf4j.Logger;
import org.slf4j.LoggerFactory;
import org.springframework.beans.factory.annotation.Value;
import org.springframework.mail.MailSendException;
import org.springframework.mail.javamail.JavaMailSender;
import org.springframework.mail.javamail.MimeMessageHelper;
import org.springframework.stereotype.Service;
import javax.mail.MessagingException;
import javax.mail.internet.InternetAddress;
import javax.mail.internet.MimeMessage;
import java.io.UnsupportedEncodingException;
@Service
public class MailService {
  private static final Logger LOGGER = LoggerFactory.getLogger(MailService.class);
  @Value("${wbs.mail.active}")
  private boolean active;
  @Value("${spring.mail.username}")
  private String fromMailAdress;
  @Value("${wbs.mail.personal}")
  private String fromMailPersonal;
  private final JavaMailSender mailSender;
  public MailService(JavaMailSender mailSender) {
    this.mailSender = mailSender;
  }
  public void send(Mail mail) throws MailConnectionException {
    if (!active) {
      return;
    }
    try {
      MimeMessage message = mailSender.createMimeMessage();
      MimeMessageHelper helper = new MimeMessageHelper(message);
      helper.setSubject(mail.getSubject());
      helper.setFrom(new InternetAddress(fromMailAdress, fromMailPersonal));
      helper.setTo(mail.getTo());
      helper.setText(mail.getText());
      LOGGER.debug("Sending E-Mail \"{}\" to {} ", mail.getSubject(), String.join(", ", mail.getTo()));
      mailSender.send(message);

      LOGGER.info("Send E-Mail \"{}\" to {} ", mail.getSubject(), String.join(", ", mail.getTo()));
    } catch (MessagingException | UnsupportedEncodingException e) {
      throw new MailDeliveryException(e, "Beim Senden der Mail {0} ist ein Fehler aufgetreten", mail.toString());
    } catch (MailSendException e) {
      throw new MailConnectionException();
    }
  }
}
\end{lstlisting}

\begin{lstlisting}[caption={Modul für die allgemeine Verarbeitung von Ausnahmen.}, label={lst:ExceptionMapper}]
package de.hsfulda.et.wbs.http.resource;
import de.hsfulda.et.wbs.core.HalJsonResource;
import de.hsfulda.et.wbs.core.WbsUser;
import de.hsfulda.et.wbs.core.exception.MailDeliveryException;
import de.hsfulda.et.wbs.core.exception.ResourceNotFoundException;
import de.hsfulda.et.wbs.core.exception.TransaktionValidationException;
import de.hsfulda.et.wbs.core.exception.ZielortLockedException;
import org.slf4j.Logger;
import org.slf4j.LoggerFactory;
import org.springframework.http.HttpStatus;
import org.springframework.http.ResponseEntity;
import org.springframework.security.core.annotation.AuthenticationPrincipal;
import org.springframework.web.bind.annotation.ControllerAdvice;
import org.springframework.web.bind.annotation.ExceptionHandler;
@ControllerAdvice
public class ExceptionMapper {
  private static final Logger LOGGER = LoggerFactory.getLogger(ExceptionMapper.class);
  @ExceptionHandler({ResourceNotFoundException.class})
  public final ResponseEntity<HalJsonResource> resourceNotFoundException(Throwable exc) {
    return toResponse(HttpStatus.NOT_FOUND, exc);
  }
  @ExceptionHandler({IllegalStateException.class, MailDeliveryException.class})
  public final ResponseEntity<HalJsonResource> illegalStateException(@AuthenticationPrincipal WbsUser user,
      Throwable exc) {
    LOGGER.error("Unerwarteter Fehler durch {}", user.getUsername());
    LOGGER.error(exc.getMessage(), exc);
    return toResponse(HttpStatus.INTERNAL_SERVER_ERROR, exc);
  }
  @ExceptionHandler({IllegalArgumentException.class})
  public final ResponseEntity<HalJsonResource> illegalArgumentException(Throwable exc) {
    return toResponse(HttpStatus.BAD_REQUEST, exc);
  }
  //...
  @ExceptionHandler(ZielortLockedException.class)
  public final ResponseEntity<HalJsonResource> zielortLockedException(Throwable exc) {
    return toResponse(HttpStatus.LOCKED, exc);
  }
  @ExceptionHandler(TransaktionValidationException.class)
  public final ResponseEntity<HalJsonResource> transaktionValidationException(Throwable exc) {
    return toResponse(HttpStatus.BAD_REQUEST, exc);
  }
  private ResponseEntity<HalJsonResource> toResponse(HttpStatus status, Throwable exc) {
    HalJsonResource message = new HalJsonResource();
    message.addProperty("message", exc.getMessage());
    return new ResponseEntity<>(message, status);
  }
}
\end{lstlisting}


