
\chapter*{Zusammenfassung}
\markboth{Abstract}{}
\label{sec:Abstract}

In dieser Arbeit wird geprüft, welche Rolle eine saubere Softwarearchitektur für eine erweiter"= und wartbare Software spielt. Dabei wird die Transaktionserfassung als Teil des Systems WBS Alarm analysiert, bewertet und überarbeitet. WBS Alarm ist eine Open Source Anwendung zur Kleiderverwaltung der freiwilligen Feuerwehr in Eschenstruth. Aus den gewonnen Erkenntnissen werden Lösungen modelliert und beurteilt, ob die empfohlenen Vorgehensweisen einen Beitrag zu wartbarer Software leisten. Dabei wird festgestellt, dass eine Restrukturierung eines Teilsystems neue Aufgaben im Gesamtsystem schafft und dieses somit bei Restrukturierung von Teilsystemen immer mit in Betracht gezogen werden sollte.

{\let\clearpage\relax\chapter*{Abstract}}

\begin{otherlanguage}{british}
This thesis examines the role of a clean software architecture for an extensible and maintainable software. Thereby the transaction capture as part of the system WBS Alarm is analyzed, evaluated and revised. WBS Alarm is an open source application for the clothing management of the voluntary fire brigade in Eschenstruth. Based on the gained knowledge, solutions are modelled and it is assessed whether the recommended procedures contribute to maintainable software. It is stated that restructuring a subsystem creates new tasks in the overall system and that this should therefore always be taken into account when restructuring subsystems.
\end{otherlanguage}